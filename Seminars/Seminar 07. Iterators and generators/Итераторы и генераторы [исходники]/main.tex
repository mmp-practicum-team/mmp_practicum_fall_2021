\documentclass[fleqn, xcolor=x11names]{beamer}
\usepackage[utf8]{inputenc}
\usetheme{MyAmsterdam} %тема\
\usecolortheme{default}
\usepackage[russian]{babel}
\usepackage[OT1]{fontenc}
\usepackage{amsmath}
\usepackage{amsfonts}
%\usepackage{amssymb}
\usepackage{hyperref}
\usepackage{graphics, graphicx}
\usepackage{color}
\usepackage{hyperref}
\usepackage{enumerate}
\usepackage{minted}
\usemintedstyle{default}

\setbeamertemplate{navigation symbols}{} % отключить навигационные значки
\setbeamertemplate{footline}{%
    \hspace{0.94\paperwidth}%
    \usebeamerfont{title in head/foot}%
    \insertframenumber\,/\,\inserttotalframenumber%
} % переопределить нижнуюю панель, убрать всё кроме номера слайда


%\usefonttheme[onlylarge]{structurebold} % названия и текст в колонтитулах выводится полужирным шрифтом.
\usefonttheme[onlymath]{serif}  % привычный шрифт для математических формул
%\setbeamerfont*{frametitle}{size=\normalsize,series=\bfseries} % шрифт заголовков слайдов
\usepackage[nopar]{lipsum} %для генерации большого текста


\newminted[pcode]{python3}{baselinestretch=1, fontsize=\small}
\newmintinline[pinline]{python3}{baselinestretch=1}

\usepackage{tikz}
\usetikzlibrary{arrows,positioning}
\usepackage{listings}
\lstset{language=Python}

\newcommand{\real}{\mathbb{R}}
\newcommand{\norm}{\mathop{\rm norm}\limits}
\newcommand{\softmax}{\mathop{\rm softmax}\limits}

\definecolor{beamer@blendedblue}{rgb}{0.037,0.366,0.75}

\title[Введение в Python]{\bfseries Занятие 7: Итераторы и генераторы}
\author[Чернышёв~А.\,В.]{Александр Чернышёв}
\subtitle{Практикум на ЭВМ 2020/2021}
\institute[ВМК МГУ]{МГУ имени М. В. Ломоносова, факультет ВМК, кафедра ММП}
\date{11 октября 2021 г.}

\begin{document}

\begin{frame}
\maketitle
\end{frame} 

\begin{frame}[fragile]\frametitle{Введение}

Сколько памяти расходуется при выполнении?

\begin{pcode}
>>> import numpy as np
>>> list_of_numbers = np.arange(0, 10 ** 8)
>>> sum = 0
>>> for item in list_of_numbers: 
...     sum += item
\end{pcode}

\end{frame} 

\begin{frame}[fragile]\frametitle{Введение}

Сколько памяти расходуется при выполнении? ($O(n)$)

\begin{pcode}
>>> n = 10 ** 8
>>> list_of_numbers = np.arange(0, n)
>>> sum_numbers = 0
>>> for item in list_of_numbers: 
...     sum_numbers += item
\end{pcode}


Существует реализация с памятью $O(1)$:

\begin{pcode}
>>> item, sum_numbers = 0, 0
>>> while item < n:
...     sum_numbers += item
...     item += 1
\end{pcode}

Ещё одна реализация с памятью $O(1)$:
\begin{pcode}
>>> sum_numbers = 0
>>> for item in range(0, n): 
...     sum_numbers += item
\end{pcode}


\end{frame} 


\section{Итераторы}
\subsection*{}

\begin{frame}[fragile]\frametitle{Семантика оператора \texttt{for}}
\begin{pcode}
>>> for elem in elem_collection:
...     do_something(elem)
\end{pcode}

\hfill

Концептуально такой цикл \pinline{for} делает следующее:
\begin{pcode}
>>> it = iter(elem_collection) # инициализация  
>>> while True:
...     try:
...         elem = next(it) # получить следующий элемент
...     except StopIteration: # если элементов больше нет
...         break
...     do_something(elem)
\end{pcode}
\end{frame}

\begin{frame}[fragile]\frametitle{Метод \texttt{\_\_iter\_\_}}
Метод \pinline{__iter__} должен вернуть экземпляр класса, который будет реализовывать протокол итераций, например \pinline{self}.

\hfill

По сути, это перегрузка встроенной обобщённой функции \pinline{iter}.

\hfill

Примеры использования функции \pinline{iter}:
\begin{pcode}
>>> my_list = [0, 1, 2]
>>> iter(my_list)
<list_iterator at 0x7f866bacdf28>
>>> my_list.__iter__()
<list_iterator at 0x7f8763166710>
>>> x = 1
>>> iter(x)
TypeError: 'int' object is not iterable
\end{pcode}
\end{frame}



\begin{frame}[fragile]\frametitle{Метод \texttt{\_\_next\_\_}}
Метод \pinline{__next__} возвращает следующий по порядку
элемент итератора. Если такого элемента нет, то метод
должен поднять исключение \pinline{StopIteration}.

Считается, что если элемент поднял \pinline{StopIteration}, то все следующие тоже поднимут это исключение!


\hfill


По сути, это перегрузка встроенной обобщённой функции \pinline{next}.
  
\hfill

Примеры использования функции \pinline{next}:
\begin{pcode}
>>> my_list = [0, 1, 2]
>>> it = iter(my_list)
>>> next(it)
0
>>> it.__next__() # то же самое, что next(it)
1
>>> next(my_list)
TypeError: 'list' object is not an iterator
\end{pcode}
\end{frame}

\begin{frame}[fragile]\frametitle{Протокол итераций и итератор}
Протокол итераций (протокол итераторов, iterator protocol) заключается в концепции итерирования по контейнерам с помощью методов \pinline{__iter__} и \pinline{__next__}

\hfill

Итерируемое (iterable) --- то, что имеет метод \pinline{__iter__}

\hfill

Итератор (iterator) --- то, что имеет методы \pinline{__iter__} и \pinline{__next__}, т.е. поддреживает протокол итераций

\end{frame}

\begin{frame}[fragile]\frametitle{Пример написания своего итератора}
\begin{pcode}
>>> class MyFile:
...    def __init__(self, file_name):
...         self.file_name = file_name
...        
...     def __iter__(self):
...         self.file = open(self.file_name, 'r')
...         return self
...     
...     def __next__(self):
...         new_line = self.file.readline()
...         new_line = new_line.lower()
...         if new_line == '':
...             self.file.close()
...             raise StopIteration
...         return new_line
\end{pcode}

Зачем это может быть нужно?
\end{frame}

\begin{frame}[fragile]\frametitle{Особенности \texttt{iter} и \texttt{next}}
Другая форма вызова \pinline{iter}: принимает функцию и значение, вызывает функцию, пока она не вернёт это значение
\begin{pcode}
>>> with open('my_file.txt', 'r') as input:
...     for line in iter(input.readline, ""):
...         print(line)
\end{pcode}

\hfill

Для функции \pinline{next} можно вторым аргументом указать значение, которое необходимо вернуть в случае исключения:
\begin{pcode}
>>> next(iter([]), 153)
153
\end{pcode}
\end{frame}

\begin{frame}[fragile]\frametitle{Протокол итераций и \texttt{\_\_getitem\_\_}}
Напоминание: \pinline{__getitem__} возвращает элемент по индексу либо поднимает \pinline{IndexError}, если элемента нет

\hfill

Если реализован метод \pinline{__getitem__}, протокол итераций будет реализован автоматически

\begin{pcode}
>>> class MyList:
...    def __init__(self, *args):
...        self.list = list(args)
...        
...    def __getitem__(self, i): 
...        return self.list[i] 
...
>>> for elem in MyList(1, 2):
...     print(elem)
1
2
\end{pcode}

Как реализован протокол итераций исходя из примера? 
\end{frame}

\begin{frame}[fragile]\frametitle{Протокол итераций и \texttt{\_\_getitem\_\_}}
Пример, когда итератор будет работать некорректно:
\begin{pcode}
>>> class MyDict:
...     def __init__(self, **kwargs):
...         self.dict = dict(kwargs)
...         
...     def __getitem__(self, i):
...         return self.dict[i]
...
>>> for elem in MyDict(a=1, b=2):
...    print(elem)
KeyError: 0
\end{pcode}
\end{frame}

\begin{frame}[fragile]\frametitle{Протокол итераций и \texttt{\_\_contains\_\_}}
Напоминание: \pinline{__contains__} возвращает \pinline{True} если переданный элемент содержится в экземпляре (перегрузка \pinline{in} и \pinline{not in})

\hfill

Если класс реализует протокол итераций, метод \pinline{__contains__} реализуется автоматически <<следующим образом>>:

\begin{pcode}
>>> class SomeClass:
...     # это реализуется автоматически
...     def __contains__(self, target_elem):
...         for elem in iter(self):
...             if target_elem == elem:
...                 return True
\end{pcode}

\hfill

Пример для \pinline{MyList}:
\begin{pcode}
>>> 2 in MyList(1, 2, 3, 4)
True
\end{pcode}

\end{frame}

\section{Генераторы}
\subsection*{}
\begin{frame}[fragile]\frametitle{Генераторы}
Генератор --- это функция, которая использует не только оператор \pinline{return}, но и оператор \pinline{yield}

\hfill

В результате выполнения оператора \pinline{yield} работа функции приостанавливается, а не прерывается, как при использовании оператора \pinline{return}

\hfill

Любой генератор --- итератор, обратное неверно


\begin{pcode}
>>> def f():
...     print("Start")
...     x = 17
...     yield x
...     yield x + 1
...     print("Done")
\end{pcode}

\end{frame}

\begin{frame}[fragile]\frametitle{Пример работы генератора}
\begin{pcode}
>>> def f():
...     print("Start")
...     x = 17
...     yield x
...     yield x + 1
...     print("Done")
\end{pcode}

\hfill

\begin{pcode}
>>> type(f)              >>> next(gen)
<class function>         18
>>> gen = f()            >>> next(gen)
>>> type(gen)            Done
<class generator>        StopIteration:
>>> next(gen)            
Start                    
17                       
\end{pcode}

\end{frame}

\begin{frame}[fragile]\frametitle{Пример генератора: unique}
\begin{pcode}
>>> def unique(iterable, seen=None):
...     seen = set(seen or [])
...     for item in iterable:
...         if item not in seen:
...             seen.add(item)
...             yield item
...
>>> sequence = [1, 1, 2, 3]
>>> unique(xs)
<generator object unique at 0x7f87609ccc50>
>>> for elem in unique(xs):
...     print(l)
1
2
3
\end{pcode}
\end{frame}

\begin{frame}[fragile]\frametitle{Пример генератора: chain}
\begin{pcode}
>>> def chain(*iterables):
...     for iterable in iterables:
...         for item in iterable:
...             yield item
\end{pcode}

\begin{pcode}
>>> xs = range(3)
>>> ys = [42]
>>> chain(xs, ys)
<generator object chain at 0x10311d708>
>>> list(chain(xs, ys))
[0, 1, 2, 42]
>>> 42 in chain(xs, ys)
True
\end{pcode}
\end{frame}

\begin{frame}[fragile]\frametitle{Переиспользование генератора}
Генераторы, как и любые итераторы, истощаются:
\begin{pcode}
>>> def f():
...     yield 42
...
>>> gen = f()
>>> list(gen)
[42]
>>> list(gen) 
[]
\end{pcode}

\hfill

Не надо переиспользовать генераторы!
\end{frame}

\begin{frame}[fragile]\frametitle{Генератор --- способ задания итератора}
\begin{pcode}
>>> def my_file(file_name):
...     file_obj = open(file_name, 'r')
...     for new_line in file_obj:
...         new_line = new_line.lower()
...         if new_line != '':
...             yield new_line
...         else:         
...             file_obj.close()
\end{pcode}
\end{frame}

\begin{frame}[fragile]\frametitle{Генераторные выражения}
%Но есть похожая конструкция
Генераторное-выражение:

\begin{pcode}
>>> gen = (x ** 2 for x in range(0, 5))
>>> gen
<generator object <genexpr> at 0x7f87609d7308>
>>> list(gen)
[0, 1, 4, 9, 16]
\end{pcode}

\hfill

%Важно: г
Генераторы списков, словарей множеств не являются генераторами!
\end{frame}


\begin{frame}[fragile]\frametitle{Оператор \pinline{yield from}}

Оператор \pinline{yield from} позволяет делегировать выполнение
другому генератору: 
\begin{pcode}
>>> def f():
...     yield 1
...     yield 2
...    
>>> def g():
...     s = f()
...     yield from s
...
>>> list(g())
[1, 2]
\end{pcode}
\end{frame}

\begin{frame}[fragile]\frametitle{Продвинутое использование генераторов}
Генераторы не так просты как кажутся

\hfill

Продвинутые методы работы с генераторами:
\begin{itemize}
\item Можно передать в генератор исключение и поднять его в месте, где генератор приостановил исполнение (\pinline{.throw})

\item Можно передавать в генератор свои значения (\pinline{.send})

\item Можно <<закрыть>> генератор, передав ему специальное исключение (\pinline{.close})
\end{itemize}

\hfill

Подробности можно узнать здесь:

\href{https://compscicenter.ru/courses/python/2015-autumn/classes/1542/}{https://compscicenter.ru/courses/python/2015-autumn/classes/1542/}

\href{http://dabeaz.com/coroutines/Coroutines.pdf}{http://dabeaz.com/coroutines/Coroutines.pdf}
\end{frame}


\section{Стандартные итераторы}
\subsection*{}
\begin{frame}[fragile]\frametitle{\texttt{range}}

\pinline{range} возвращает итерируемый объект \texttt{range object} (не~итератор!)

\begin{pcode}
>>> my_range = range(0, 5)
>>> my_range
range(0, 5)
>>> next(my_range)
TypeError: 'range' object is not an iterator
>>> iter(my_range)
<range_iterator at 0x7f8780179240>
>>> next(iter(my_range))
0
\end{pcode}

\pinline{range} при любых аргументах требует константный объём памяти

\hfill

В Python2 аналог \pinline{range} --- \pinline{xrange}
\end{frame}

\begin{frame}[fragile]\frametitle{\texttt{zip}}

\pinline{zip} возвращает итератор

\begin{pcode}
>>> my_list_1 = [1, 2]
>>> my_list_2 = ['a', 'b']
>>> my_tuples = zip(my_list_1, my_list_2)
>>> my_tuples
<zip at 0x7f8769928e08>
>>> iter(my_tuples)
<zip at 0x7f8769928e08>
>>> next(my_tuples)
(1, 'a')
>>> next(my_tuples)
(2, 'b')
>>> next(my_tuples)
StopIteration: 
\end{pcode}

\end{frame}

\begin{frame}[fragile]\frametitle{\texttt{enumerate}}
\pinline{enumerate} возвращает итератор

\hfill
\begin{pcode}
>>> my_list = ['a', 'b', 'c']
>>> it = enumerate(my_list)
>>> it
<enumerate at 0x7f87609c6360>
>>> next(it)
(0, 'a')
\end{pcode}
%\hfill

%Как выглядел бы enumerate, если бы мы его писали сами?
\end{frame}


\begin{frame}[fragile]\frametitle{\texttt{map}}
\pinline{map(func, sequence1, sequence2 ... )} возвращает итератор, состоящий из результатов применений функции \pinline{func} к элементам \pinline{sequence1}, \pinline{sequence2} 

\begin{itemize}
\item \pinline{func} --- функция, которая применяется к каждому элементу \pinline{sequence}
\item \pinline{sequence1}, \pinline{sequence2} \ldots  --- итераторы
\end{itemize}

\begin{pcode}
>>> it = map(lambda x: x * 2, [0, 1, 2, 3])
>>> list(it)
[0, 1, 4, 9]
\end{pcode}

\begin{pcode}
>>> it = map(lambda x, y: x + y, [0, 1, 2, 3], [10, 100, 500])
>>> list(it)
[10, 101, 502]
\end{pcode}

\end{frame}

\begin{frame}[fragile]\frametitle{\texttt{filter}}
\pinline{filter(func, sequence)} возвращает итератор, состоящий из элементов \pinline{sequence}, для которых \pinline{func} вернула \pinline{True}

\begin{pcode}
>>> it = filter(lambda x: x > 0, [0, -1, 2, -3, 5])
>>> list(it)
[2, 5]
\end{pcode}

\hfill

И \pinline{filter}, и \pinline{map} можно проще записать через генераторы списков!
\end{frame}

\begin{frame}[fragile]\frametitle{\texttt{reduce}}

\pinline{reduce(func, sequence)} --- кумулятивное применение функции \pinline{func} к элементам последовательности \pinline{sequence}

\begin{pcode}
>>> from functools import reduce
...
>>> # ((1*2)*3)*4
>>> it = reduce(lambda x, y: x * y, [1, 2, 3, 4]) 
>>> it
24 
>>> it = reduce(lambda x, y: '({}*{})'.format(str(x), str(y)),
                [1, 2, 3, 4])
(((1*2)*3)*4)
\end{pcode}
\end{frame}


\section{Модуль itertools}
\subsection*{}

\begin{frame}[fragile]\frametitle{Соединение и повторение итераторов}
Написанная нами \pinline{chain} есть в \pinline{itertools}:
\begin{pcode}
>>> from itertools import chain
>>> it1 = iter([1,2,3])
>>> it2 = iter([4,5])
>>> list(chain(it1, it2))
[1, 2, 3, 4, 5]
\end{pcode}

\pinline{repeat} для создания итератора повторяющейся последовательности:
\begin{pcode}
>>> from itertools import repeat
>>> b = []
>>> for i in repeat([1, 2], 4):
...     b.append(i)
>>> print (b)
[1, 2, 1, 2, 1, 2, 1, 2]
\end{pcode}
\end{frame}

\begin{frame}[fragile]\frametitle{Бесконечные итераторы}

\begin{pcode}
from itertools import count # бесконечный range
c = []
for i in count(0, 5):
    c.append(i)
    if i > 20:
        break
print(c)
[0, 5, 10, 15, 20, 25]
\end{pcode}

\begin{pcode}
>>> from itertools import cycle # бесконечный повтор 
>>> d = []                      # последовательности
>>> it = cycle([1,2,3])
>>> for i, j in enumerate(it):
...     if i > 6:
...         break
...     d.append(j)
>>> print(d
[1, 2, 3, 1, 2, 3, 1]
\end{pcode}
\end{frame}

\begin{frame}[fragile]\frametitle{Срезы и взятие частей}
Срез для произвольного итератора:
\begin{pcode}
>>> from itertools import islice  
>>> e = islice(range(10), 0, 8, 2) 
>>> list(e)
[0, 2, 4, 6]
\end{pcode}

\hfill

Выбрасывать из итератора элементы, пока не найдётся элемент, удовлетворяющий условию:
\begin{pcode}
>>> from itertools import dropwhile
>>> f = dropwhile(lambda x: x < 5, range(10))
>>> print (list(f))
[5, 6, 7, 8, 9]
\end{pcode}

\end{frame}

\begin{frame}[fragile]\frametitle{Комбинаторные итераторы}
\begin{pcode}
>>> from itertools import permutations
>>> it = permutations("YN")
>>> print (list(it))
[('Y', 'N'), ('N', 'Y')]
\end{pcode}

\hfill

\begin{pcode}
>>> from itertools import combinations
>>> it = combinations("ABC", 2)
>>> print (list(it))
[('A', 'B'), ('A', 'C'), ('B', 'C')]
\end{pcode}

\hfill

\begin{pcode}
>>> from itertools import product
>>> it = product("AB", repeat=2)
>>> print (list(it))
[('A', 'A'), ('A', 'B'), ('B', 'A'), ('B', 'B')]
\end{pcode}
\end{frame}

\section*{}

\begin{frame}[fragile]\frametitle{Итераторы в задачах с данными большого объёма}
Итераторы могут быть полезны в ситуациях, когда хранить всю выборку в памяти невозможно

\hfill

Алгоритм работы:
\begin{enumerate}
\item Сохранить данные на жёстком диске в хорошем формате
\item Написать итератор, считывающий данные неполностью и преобразующий их в вектора
\item Обучать алгоритм по батчам данных
\end{enumerate}

%\hfill
%
%Некоторые библиотеки поддерживают работу с итераторами из коробки (например, Gensim).
\end{frame}

\begin{frame}{Варианты применения итераторов в data science}
\begin{itemize}
    \item Работа с огромной текстовой коллекцией (коллекцией изображений, и т.д.)
    
    \item Изменение порядка подачи объектов на обучение
    
    \item Быстрая генерация из сложных объектов более простых (например, по тексту генерируем пары предложений)
\end{itemize}
\end{frame}

\begin{frame}[fragile]\frametitle{Заключение}
\begin{itemize}
\item Итераторы позволяют не хранить одновременно в памяти все значения последовательности

\item Итераторы можно задавать с помощью описания специальных методов \pinline{__iter__} и \pinline{__next__} или декларативно с помощью генераторов

\item Многие операции с операторами уже реализованы в библиотеке \pinline{itertools} либо средствами \pinline{itertools} их можно реализовать проще

\item Итераторы необходимо применять тогда, когда невозможно хранить весь объём данных в памяти
\end{itemize}
\end{frame}




\end{document}